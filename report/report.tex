\documentclass[twoside,twocolumn]{article}

\usepackage{abstract}
\renewcommand{\abstractnamefont}{\normalfont\bfseries} % Set the "Abstract" text to bold
\renewcommand{\abstracttextfont}{\normalfont\small\itshape} % Set the abstract itself to small italic text

\usepackage{hyperref} % For hyperlinks in the PDF

\usepackage{titling} % Customizing the title section
\usepackage{titlesec} % Allows customization of titles
\usepackage{amssymb}
\usepackage{amsmath}
\usepackage{amsthm}
\newtheorem{remark}{Remark}
\newtheorem{definition}{Definiton}
\newcommand\given[1][]{\:#1\vert\:}

%----------------------------------------------------------------------------------------
%	TITLE SECTION
%----------------------------------------------------------------------------------------

\setlength{\droptitle}{-4\baselineskip} % Move the title up

\pretitle{\begin{center}\Huge\bfseries} % Article title formatting
\posttitle{\end{center}} % Article title closing formatting
\title{Sample-Based control of Discrete Time Hybrid Systems with Non-Gaussian Noise} % Article title
\author{%
\textsc{Luke Rickard}\thanks{A thank you or further information} \\[1ex] % Your name
\normalsize University of Oxford \\ % Your institution
\normalsize \href{mailto:rickard@robots.ox.ac.uk}{rickard@robots.ox.ac.uk} % Your email address
\and % Uncomment if 2 authors are required, duplicate these 4 lines if more
\textsc{Licio Romao}\thanks{Corresponding author} \\[1ex] % Second author's name
\normalsize University of Oxford \\ % Second author's institution
\normalsize \href{mailto:licio.romao@cs.ox.ac.uk}{licio.romao@cs.ox.ac.uk} % Second author's email address
\and % Uncomment if 2 authors are required, duplicate these 4 lines if more
\textsc{Alessandro Abate}\thanks{Corresponding author} \\[1ex] % Second author's name
\normalsize University of Oxford \\ % Second author's institution
\normalsize \href{mailto:alessandro.abate@cs.ox.ac.uk}{alessandro.abate@cs.ox.ac.uk} % Second author's email address
}
\date{\today} % Leave empty to omit a date
\renewcommand{\maketitlehookd}{%
\begin{abstract}
\noindent Hello there % Dummy abstract text - replace \blindtext with your abstract text
\end{abstract}
}

%----------------------------------------------------------------------------------------


\begin{document}
\maketitle

\section{Introduction}
Many complex systems can be modelled as \emph{stochastic hybrid systems}. Such systems consist of continuous and discrete time processes where the evolution of the continuous system can depend on the discrete process. The discrete process can be considered as defining different \emph{modes} of operation of the continuous system. We consider such hybrid systems in the case where the discrete process can be controlled (or steered) and where there is no control over the evolution of this discrete process (unsteered). In both cases we allow for a continuous control input at all timesteps.

In order to use these systems we are interested in designing some control input such that we can satisfy some property. Of particular interest in this work is the reach-avoid property. This property requires a controller which can \emph{reach} some goal states while \emph{avoiding} some unsafe states. For example, in the case of unmanned aerial vehicle (UAV) control, we might want to fly to a given destination whilst avoiding fixed obstacles.

In many systems (especially those which are safety-critical), the design of such control inputs requires some consideration to underlying randomness. Here we model this randomness as \emph{process noise}, which appears as an additive term in the system dynamics. This process noise can vary between the different modes of the stochastic hybrid system. One possible source of this noise (in our UAV case) is turbulence and/ or wind gusts. By using a hybrid system we can thus consider modes with identical dynamics but differing levels wind speeds.

In addition to this process noise, we consider uncertainty in the transition between modes. This allows for the design of controllers which can be robust to a changing mode of operation (in the unsteered case). As well as those that can optimally choose a mode of operation (in the steered case). In our UAV example, this means that we can design a controller which is robust to a changing wind speed, but is able to take sensible actions in low wind conditions.

The underlying process noise is addressed using \emph{probably approximately correct} (PAC) guarantees, which does not require any knowledge about the underlying distribution.
The hybrid switching is treated as having a known probabillity, and thus is modelled using a \emph{Markov chain} (MC)(in the unsteered case) or \emph{Markov Decision Process} (MDP)(for the steered case).

\subsection{Contributions}

In this work we extend the work of \cite{} to the case of state-indepenedent stochastic hybrid systems, in doing so we make the methods applicable to a much broader range of systems. We also provide some case studies later to demonstrate the effectiveness of our methods on these complex systems.

\subsection{Related Works}

\section{Responsible Research and Innovation}

\subsection{Potential for misuse}

The methods proposed within this work are very general and could be applied to any hybrid system. As such the potential for misuse is very large since there are many hybrid systems that can be used for malicious purposes. However, misuse of our work is unlikely to allow for the control of systems which cannot be currently controlled (even if suboptimally) and as such we believe the negative impacts of misuse are limited in scope.

\subsection{Safety considerations}

Our work generally deals with safety, and in particular in the safe control of hybrid systems with unkown noise distributions. As such the adoption of our work could lead to an increase in safety for those who interact with such systems (which we argue is a very large population). If misused our work does offer the possibility of increased safety for actors with ill-intent, however we do not believe that increased safety of such actors can be necessarily argued as unethical. Furthermore, defining who has such ill-intent is not straightforward and is not a problem we tackle here. Because of these points we make this work freely available to all.

\subsection{Consequences and Legal Responsibility}
\label{subsec:legal}

Since we can offer probabilistic guarantees on safety our guarantees should not be blindly trusted. Where possible some oversight should be provided to ensure systems do not enter unsafe operating conditions. In any case, the probabilistic guaranees should be followed through to gain some knowledge of the risk of harm to people, if this is of an unacceptable level a tighter confidence bound can be chosen to reduce the risk (at the cost of more computation). We would therefore place legal responsibility on the person(s) implementing our system as they would be most well equipped to understand these risks.

\subsection{Data Privacy}

We do not make much considerations towards data privacy here since we do not make use of any personal data.

\subsection{Acceptance/ Use by Different Societal Groups}

Since our proposed methods are so general they have the potential to be taken up by many different societal, ethnic and gender groups for whatever purposes they see fit.

\subsection{Job Losses}

This work has the potential to lead to job losses, since the control of hybrid systems can (occasionally and suboptimally) be achieved manually by human operators. This work is better achieved by automation, and it is our view that ongoing automation should not be seen as a negative. Instead, we believe that an eventual goal should be automation of all jobs, with the only negatives of this being attributable to the larger economic system.

\subsection{Overview}

In summary, we consider that the publication of this work is responsible and ethical. The potential for increased safety of many systems outweighs the potential negative implications of misuse. However, in order to ensure that our proposed method is used correctly we would always encourage taking the precautions outlined in \autoref{subsec:legal}.

\section{Foundations}

\subsection{Definitions}

\begin{definition}[Linear Time Invariant Dynamical System]
	\begin{equation}
		\begin{aligned}
			x(k+1)&=Ax(k)+Bu(k)+q+w\\
			x &\in \mathbb{R}^n\\
			u &\in \mathbb{R}^p\\
			A &\in \mathbb{R}^{n \times n}\\
			B &\in \mathbb{R}^{n \times p}\\
			q &\in \mathbb{R}^n\\
			w &\in \mathbb{R}^n\\
			k &\in \mathbb{N}
		\end{aligned}
	\end{equation}
\end{definition}

\begin{definition}[Discrete Time Hybrid System (DTHS)]

\end{definition}


\begin{remark}[Set of Dynamics]

\end{remark}

\begin{remark}[Special cases]

\end{remark}

\begin{definition}[iMDP]

\end{definition}

\begin{remark}[Product of iMDPs]
	Let $\mathcal{M}^1_\mathbb{I}=(S_1,A_1,s_1,\mathcal{P}_1)$ and $\mathcal{M}^2_\mathbb{I}=(S_2,A_2,s_2,\mathcal{P}_2)$ be two independent iMDPs. If we define the joint state and action spaces $S_\times=(S_1,S_2)$, $A_\times=(A_1,A_2)$ then we can define the joint interval probabilities as
\begin{equation}
	\begin{aligned}
	&\mathbb{P}(s'_1,s'_2 \given s_1,s_2,a_1,a_2) \in \\
	[&\underline{P_1}(s'_1 \given s_1,a_1) \cdot \underline{P_2}(s'_2\given s_2,a_2),\\
	 &\overline{P_1}(s'_1 \given s_1,a_1) \cdot \overline{P_2}(s'_2 \given s_2,a_2)]
	\end{aligned}
\end{equation}
\end{remark}


\section{Case Studies}

\section{Concluding Remarks and Future Work}
now some text


\appendix
\section{Proofs}

\begin{equation}
	\begin{aligned}
	&\mathbb{P}(s'_1,s'_2 \given s_1,s_2,a_1,a_2)\\
	&=\mathbb{P}(s'_1 \given s_1,s_2,a_1,a_2) \cdot \mathbb{P}(s'_2 \given s_1,s_2,a_1,a_2)\\
	&=\mathbb{P}(s'_1 \given s_1, a_1) \cdot \mathbb{P}(s'_2 \given s_2, a_2)\\
	 &\mathbb{P}(s'_1 \given s_1, a_1)\\
	 &\in [\underline{P_1}(s'_1 \given s_1, a_1), \overline{P_1}(s'_1 \given s_1,a_1)]\\
	 &\mathbb{P}(s'_2 \given s_2, a_2)\\
	 &\in [\underline{P_2}(s'_2 \given s_2, a_2), \overline{P_2}(s'_2 \given s_2,a_2)]\\
	 &a \in [b, c]\\
	 &d \in [e, f]\\
	 &a > b, d > e\\
	 &ad > bd, bd > be\\
	 &ad > be\\
	 &a < c, d < f\\
	 &ad < cd, cd < cf\\
	 &ad < cf\\
	 &ad \in [be, cf]\\
	 &\therefore \mathbb{P}(s'_1,s'_2 \given s_1, s_2, a_1, a_2) \\
	 &\in [\underline{P_1}(s'_1 \given s_1,a_1) \cdot \underline{P_2}(s'_2 \given s_2,a_2),\\
	 & \overline{P_1}(s'_1 \given s_1, a_1) \cdot \overline{P_2}(s'_2 \given s_2, a_2)
	\end{aligned}
\end{equation}
\end{document}
